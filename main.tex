\documentclass[12pt,a4paper]{article}
\usepackage[utf8]{inputenc}
\usepackage{dsfont} 
\usepackage[polish]{babel}
\usepackage{amsmath}
\usepackage{graphicx}
\usepackage[top=1in, bottom=1.5in, left=1.25in, right=1.25in]{geometry}

\usepackage{subfig}
\usepackage{multirow}
\usepackage{multicol}
\graphicspath{{Imagens/}}
\usepackage{xcolor,colortbl}
\usepackage{float}

\newcommand \comment[1]{\textbf{\textcolor{red}{#1}}}

%\usepackage{float}
\usepackage{fancyhdr} % Required for custom headers
\usepackage{lastpage} % Required to determine the last page for the footer
\usepackage{extramarks} % Required for headers and footers
\usepackage{indentfirst}
\usepackage{placeins}
\usepackage{scalefnt}
\usepackage{xcolor,listings}
\usepackage{textcomp}
\usepackage{color}
\usepackage{verbatim}
\usepackage{framed}

\definecolor{codegreen}{rgb}{0,0.6,0}
\definecolor{codegray}{rgb}{0.5,0.5,0.5}
\definecolor{codepurple}{HTML}{C42043}
\definecolor{backcolour}{HTML}{F2F2F2}
\definecolor{bookColor}{cmyk}{0,0,0,0.90}  
\color{bookColor}

\lstset{upquote=true}

\lstdefinestyle{mystyle}{
	backgroundcolor=\color{backcolour},   
	commentstyle=\color{codegreen},
	keywordstyle=\color{codepurple},
	numberstyle=\numberstyle,
	stringstyle=\color{codepurple},
	basicstyle=\footnotesize\ttfamily,
	breakatwhitespace=false,
	breaklines=true,
	captionpos=b,
	keepspaces=true,
	numbers=left,
	numbersep=10pt,
	showspaces=false,
	showstringspaces=false,
	showtabs=false,
}
\lstset{style=mystyle}

\newcommand\numberstyle[1]{%
	\footnotesize
	\color{codegray}%
	\ttfamily
	\ifnum#1<10 0\fi#1 |%
}

\definecolor{shadecolor}{HTML}{F2F2F2}

\newenvironment{sqltable}%
{\snugshade\verbatim}%
{\endverbatim\endsnugshade}

% Margins
\addtolength{\footskip}{0cm}
\addtolength{\textwidth}{1.4cm}
\addtolength{\oddsidemargin}{-.7cm}

\addtolength{\textheight}{1.6cm}
%\addtolength{\topmargin}{-2cm}

% paragrafo
\addtolength{\parskip}{.2cm}

% Set up the header and footer
\pagestyle{fancy}
\rhead{\hmwkAuthorName} % Top left header
\lhead{\hmwkClass: \hmwkTitle} % Top center header
\rhead{\firstxmark} % Top right header
\lfoot{Dawid Kalinowski} % Bottom left footer
\cfoot{} % Bottom center footer
\rfoot{} % Bottom right footer
\renewcommand{\headrulewidth}{1pt}
\renewcommand{\footrulewidth}{1pt}

    
\newcommand{\hmwkTitle}{Aplikacja do gry w Gomoku} % Tytuł projektu
\newcommand{\hmwkDueDate}{\today} % Data 
\newcommand{\hmwkClass}{Technologie chmurowe} % Nazwa przedmiotu
\newcommand{\hmwkAuthorName}{Dawid Kalinowski} % Imię i nazwisko

% trabalho 
\begin{document}
% capa
\begin{titlepage}
    \vfill
	\begin{center}
	\hspace*{-1cm}
	\vspace*{0.5cm}
    \includegraphics[scale=0.55]{imagens/loga.png}\\
	\textbf{Uniwersytet Gdański \\ [0.05cm]Wydział Matematyki, Fizyki i Informatyki \\ [0.05cm] Instytut Informatyki}

	\vspace{0.6cm}
	\vspace{4cm}
	{\huge \textbf{\hmwkTitle}}\vspace{8mm}
	
	{\large \textbf{\hmwkAuthorName}}\\[3cm]
	
		\hspace{.45\textwidth} %posiciona a minipage
	   \begin{minipage}{.5\textwidth}
	   Projekt z przedmiotu technologie chmurowe na kierunku informatyka profil praktyczny na Uniwersytecie Gdańskim.\\[0.1cm]
	  \end{minipage}
	  \vfill
	%\vspace{2cm}
	
	\textbf{Gdańsk}
	
	\textbf{\hmwkDueDate}
	\end{center}
	
\end{titlepage}

\newpage
\setcounter{secnumdepth}{5}
\tableofcontents
\newpage

\section{Opis projektu}
\label{sec:Project}

Firma zajmująca się grami planszowymi, Board Games Interactive, zleciła stworzenie aplikacji webowej, która będzie docelowo obsługiwała rozgrywki w przeróżnych grach planszowych, między graczami z całego świata. Na sam początek przedsiębiorstwo postanowiło zapewnić graczom możliwość gry w grę "Gomoku". Serwis ma działać w przeglądarce, zarówno na komputerach, jak i na urządzeniach mobilnych. System musi zapewnić płynną rozgrywkę dla tysięcy graczy jednocześnie.

\subsection{Opis architektury - 6 pkt}
\label{sec:introduction}
Architektura aplikacji została zaimplementowana w środowisku Kubernetes, co w przyszłości umożliwi łatwiejsze skalowanie aplikacji oraz wdrażanie nowych zmian. Składa się z trzech oddzielnych komponentów: Backendu w Node.js, Frontendu w React.js oraz bazy danych MongoDB. Każdy z nich jest osobnym kontenerem w klastrze Kubernetes.

\subsection{Opis infrastruktury - 6 pkt}
\label{sec:Users}
Aplikacja korzysta z platformy Kubernetes, oraz przechowuje obrazy Dockerowe na platformie DockerHub. Frontend i backend używają LoadBalancer do komunikowania się z innymi klastrami w Kubernetesie.\\
Backend używa Node.js, biblioteki Express, komunikuje się z bazą danych za pomocą Mongoose. Używa również WebSocketów (biblioteka ws) oraz JWT.\\
Frontend używa frameworka React.js, a zarządzanie sesją użytkownika odbywa się za pomocą js-cookie.\\
Baza danych używa ClusterIP do komunikowania się z innymi klastrami w Kubernetesie.

\subsection{Opis komponentów aplikacji - 6 pkt}
\label{sec:FunctionalConditions}
Aplikacja składa się z trzech głównych komponentów:
\begin{itemize}
    \item \textbf{Backend (Node.js):} Backend aplikacji został wyprodukowany w Node.js przy użyciu frameworka Express.js. Odpowiada za obsługę użytkowników oraz gier, między innymi poprzez pobieranie oraz wysyłanie informacji z / do bazy danych w MongoDB. Backend udostępnia RESTful API, które obsługuje operacje takie jak rejestracja, logowanie użytkowników, tworzenie, dołączanie do gier, w przypadku Gomoku również stawianie pionków na planszy. Panel administratora udostępnia również takie funkcjonalności, jak edycja poszczególnych instancji gier, manipulowanie uprawnieniami oraz danymi zwykłych użytkowników. Do zarządzania stanem sesji użytkowników zastosowano tokeny JWT, a logowanie użytkowników jest zabezpieczone funkcją hahsującą, a dokładniej bcrypt. Aplikacja umożliwia również czatowanie z innymi użytkownikami na wspólnym forum, za pomocą protokołu MQTT.

    
    \item \textbf{Frontend (React.js):} Frontend aplikacji został zaimplementowany w bibliotece React.js. Użytkownik może, korzystając z intuicyjnych komponentów, takich jak pola tekstowe, przyciski, poruszać się po aplikacji, w szczególności dołączać do gier oraz grać. Aplikacja posiada przejrzystą planszę do gry w Gomoku, która po zmianie stanu planszy na bieżąco się odświeża. Frontend komunikuje się z Backendem za pomocą zapytań HTTP, przesyłając i odbierając dane w formacie JSON.
    Aplikacja jest również przystosowana pod urządzenia mobilne, uzależniając rozmiar poszczególnych komponentów od maksymalnej rozdzielcości okna.
    \item \textbf{Baza danych MongoDB:} Do przechowywania informacji o wszystkich użytkownikach, oraz toczonych grach w aplikacji, wykorzystano MongoDB, bazę danych NoSQL. Baza przechowuje dane użytkowników, informacje zarówno o wszystkich aktualnie toczonych meczach, jak i historycznych gier.
\end{itemize}
Każdy komponent jest skonfigurowany w osobnym pliku YAML dla Kubernetes i jest wdrażany jako oddzielny pod.

\subsection{Konfiguracja i zarządzanie - 4 pkt}
\label{sec:NonFunctionalConditions}
Konfiguracja aplikacji odbywa się poprzez pliki YAML dla Kubernetes, które definiują deploymenty i serwisy. Deployment zarówno backendu, jak i frontendu, definiuje obraz Docker, m.in. liczbę replik - w tym przypadku 3 - oraz zmienne środowiskowe, w tym URI do bazy danych MongoDB. Usługa zapewnia zewnętrzny dostęp do backendu i frontendu poprzez LoadBalancer. Baza danych również jest w 3 replikach, a komunikacja odbywa się za pomocą ClusterIP.

\subsection{Zarządzanie błędami - 2 pkt}
Aplikacja obsługuje błędy róznych zapytań, i wszelkie nieudane / nieprawidłowe działania wypisuje w konsoli użytkownika.
\label{sec:ERD} 

\subsection{Skalowalność - 4 pkt}
\label{sec:ExamplesSection}
Za skalowalność aplikacji jest odpowiedzialny Horizontal Pod Autoscaler (HPA). Jest skonfigurowany do monitorowania backendu, i zmienia liczbę jego replik w zakresie liczb 3-10, w zależności od obciążenia procesora. Aktualnie jego średnie zużycie jest ustawione na poziomie 60%.

\subsection{Wymagania dotyczące zasobów - 2 pkt}
\label{sec:ExampleTables}
Wymagania dotyczące zasobów są następujące:
\begin{itemize}
    \item \textbf{Backend:} Minimalne: 0.5 CPU, 512MB RAM, Maksymalne: 1 CPU, 1 GB RAM
    \item \textbf{Frontend:} Minimalne: 0.25 CPU, 256MB RAM, Maksymalne: 0.5 CPU, 512MB RAM
    \item \textbf{Baza danych:} Minimalne: 0.5 CPU, 1GB Ram,
    Maksymalne: 1CPU, 2GB Ram
\end{itemize}
Wszystkie powinny wykonywać się w akceptowalnym dla użytkownika czasie, poniżej 100 ms.

\subsection{Architektura sieciowa - 4 pkt}
\label{sec:ExampleResults}
Wszystkie serwisy w klastrze komunikują się ze sobą w sieci lokalnej localhost. Serwisy są skonfigurowane na działanie w następujących portach:\\
- Backend nasłuchuje na porcie 3003\\
- Frontnend na porcie 3000\\
- Baza danych na porcie 27017
\noindent

\bibliographystyle{amsplain}
\bibliography{references.bib}
\nocite{*}

\end{document}
